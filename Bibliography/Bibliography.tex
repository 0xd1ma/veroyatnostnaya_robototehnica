\documentclass[10pt,a4paper]{article}
\usepackage[utf8]{inputenc}
\usepackage[english,russian]{babel}
\usepackage{cmap}
\usepackage[OT1]{fontenc}
\usepackage{amsmath}
\usepackage{amsfonts}
\usepackage{amssymb}
\usepackage{graphicx}
\usepackage{float}
\usepackage{wrapfig}
\usepackage{caption}
\DeclareCaptionLabelSeparator{dot}{. }
\captionsetup{justification=centering,labelsep=dot}
\graphicspath{{pictures/}}
\DeclareGraphicsExtensions{.pdf,.png,.jpg,.eps}
\begin{document}

\textbf{Библиография}\\

Aberdeen, D. 2002. A survey of approximate methods for solving partially observable
Markov decision processes. Technical report, Australia National University.

Acar, E.U., and H. Choset. 2002. Sensor-based coverage of unknown environments.
\textit{International Journal of Robotic Research} 21:345–366.

Acar, E.U., H. Choset, Y. Zhang, and M.J. Schervish. 2003. Path planning for robotic
demining: Robust sensor-based coverage of unstructured environments and probabilistic
methods. \textit{International Journal of Robotic Research} 22:441–466.

Albers, S., and M.R. Henzinger. 2000. Exploring unknown environments. \textit{SIAM
Journal on Computing} 29:1164–1188.

Anguelov, D., R. Biswas, D. Koller, B. Limketkai, S. Sanner, and S. Thrun. 2002. Learning
hierarchical object maps of non-stationary environments with mobile robots. In
\textit{Proceedings of the 17th Annual Conference on Uncertainty in AI (UAI)}.

Anguelov, D., D. Koller, E. Parker, and S. Thrun. 2004. Detecting and modeling doors
with mobile robots. In \textit{Proceedings of the International Conference on Robotics and
Automation (ICRA)}.

Apostolopoulos, D., L. Pedersen, B. Shamah, K. Shillcutt, M.D. Wagner, and W.R.
Whittaker. 2001. Robotic antarctic meteorite search: Outcomes. In \textit{Proceedings of
the International Conference on Robotics and Automation (ICRA)}, pp. 4174–4179.

Araneda, A. 2003. Statistical inference in mapping and localization for a mobile
robot. In J. M. Bernardo, M.J. Bayarri, J.O. Berger, A. P. Dawid, D. Heckerman,
A.F.M. Smith, and M.West (eds.), \textit{Bayesian Statistics 7}. Oxford, UK: Oxford University
Press.

Arkin, R. 1998. \textit{Behavior-Based Robotics}. Cambridge, MA: MIT Press.

Arras, K.O., and S.J Vestli. 1998. Hybrid, high-precision localisation for the mail distributing
mobile robot system MOPS. In \textit{Proceedings of the International Conference
on Robotics and Automation (ICRA)}.

Astrom, K.J. 1965. Optimal control of Markov decision processes with incomplete
state estimation. \textit{Journal of Mathematical Analysis and Applications} 10:174–205.

Austin, D.J., and P. Jensfelt. 2000. Using multiple Gaussian hypotheses to represent
probability-distributions for mobile robots. In \textit{Proceedings of the IEEE International
Conference on Robotics and Automation (ICRA)}.

Avots, D., E. Lim, R. Thibaux, and S. Thrun. 2002. A probabilistic technique for simultaneous
localization and door state estimation with mobile robots in dynamic
environments. In \textit{Proceedings of the IEEE/RSJ Int. Conf. on Intelligent Robots and Systems
(IROS)}.

B, Triggs, McLauchlan P, Hartley R, and Fitzgibbon A. 2000. Bundle adjustment
– A modern synthesis. In W. Triggs, A. Zisserman, and R. Szeliski (eds.), \textit{Vision
Algorithms: Theory and Practice}, LNCS, pp. 298–375. Springer Verlag.

Bagnell, J., and J. Schneider. 2001. Autonomous helicopter control using reinforcement
learning policy search methods. In\textit{ Proceedings of the International Conference
on Robotics and Automation (ICRA)}.

Bailey, T. 2002. \textit{Mobile Robot Localisation and Mapping in Extensive Outdoor Environments.}
PhD thesis, University of Sydney, Sydney, NSW, Australia.

Baker, C., A. Morris, D. Ferguson, S. Thayer, C. Whittaker, Z. Omohundro, C. Reverte,
W. Whittaker, D. Hähnel, and S. Thrun. 2004. A campaign in autonomous mine
mapping. In \textit{Proceedings of the International Conference on Robotics and Automation
(ICRA)}.

Ballard, R.D. 1987. The \textit{Discovery of the Titanic}. New York, NY:Warner/Madison Press.

Bapna, D., E. Rollins, J. Murphy, M. Maimone, W.L. Whittaker, and D. Wettergreen.
1998. The Atacama Desert trek: Outcomes. In \textit{Proceedings of the International Conference
on Robotics and Automation (ICRA)}, volume 1, pp. 597–604.

Bar-Shalom, Y., and T.E. Fortmann. 1988. \textit{Tracking and Data Association.} Academic
Press.

Bar-Shalom, Y., and X.-R. Li. 1998. \textit{Estimation and Tracking: Principles, Techniques, and
Software.} Danvers, MA: YBS.

Bardi, M., Parthasarathym T., and T.E.S. Raghavan. 1999. \textit{Stochastic and Differential
Games: Theory and Numerical Methods.} Boston: Birkhauser.

Bares, J., and D.Wettergreen. 1999. Dante II: Technical description, results and lessons
learned. \textit{International Journal of Robotics Research} 18:621–649.

Barniv, Y. 1990. Dynamic programming algorithm for detecting dim moving targets.
In Y. Bar-Shalom (ed.), \textit{Multitarget-Multisensor Tracking: Advanced Applications}, pp.
85–154. Boston: Artech House.

Barto, A.G., S.J. Bradtke, and S.P. Singh. 1991. Real-time learning and control using
asynchronous dynamic programming. Technical Report COINS 91-57, Department
of Computer Science, University of Massachusetts, MA.

Batalin, M., and G. Sukhatme. 2003. Efficient exploration without localization. In
\textit{Proceedings of the International Conference on Robotics and Automation (ICRA)}.

Baxter, J., L. Weaver, and P. Bartlett. 2001. Infinite-horizon gradient-based policy
search: II. Gradient ascent algorithms and experiments. \textit{Journal of Artificial Intelligence
Research.} To appear.

Bekker, G. 1956. \textit{Theory of Land Locomotion.} University of Michigan.

Bekker, G. 1969. \textit{Introduction to Terrain-Vehicle Systems.} University of Michigan.

Bellman, R.E. 1957. \textit{Dynamic Programming.} Princeton, NJ: Princeton University Press.

Berry, D., and B. Fristedt. 1985. \textit{Bandit Problems: Sequential Allocation of Experiments.}
Chapman and Hall.

Bertsekas, Dimitri P., and John N. Tsitsiklis. 1996. \textit{Neuro-Dynamic Programming.} Belmont,
MA: Athena Scientific.

Besl, P., and N. McKay. 1992. A method for registration of 3d shapes. \textit{Transactions on
Pattern Analysis and Machine Intelligence} 14:239–256.

Betgé-Brezetz, S., R. Chatila, and M. Devy. 1995. Object-based modelling and localization
in natural environments. In \textit{Proceedings of the International Conference on
Robotics and Automation (ICRA)}.

Betgé-Brezetz, S., P. Hébert, R. Chatila, and M. Devy. 1996. Uncertain map making in
natural environments. In \textit{Proceedings of the IEEE International Conference on Robotics
and Automation (ICRA)}, Minneapolis.

Betke, M., and K. Gurvits. 1994. Mobile robot localization using landmarks. In
\textit{Proceedings of the IEEE International Conference on Robotics and Automation (ICRA)},
pp. 135–4142.

Biswas, R., B. Limketkai, S. Sanner, and S. Thrun. 2002. Towards object mapping
in dynamic environments with mobile robots. In \textit{Proceedings of the IEEE/RSJ
Int. Conf. on Intelligent Robots and Systems (IROS)}.

Blackwell, D. 1947. Conditional expectation and unbiased sequential estimation.
\textit{Annals of Mathematical Statistics} 18:105–110.

Blahut, R.E.,W. Miller, and C.H.Wilcox. 1991. \textit{Radar and Sonar: Parts I\&II.} New York,
NY: Springer-Verlag.

Borenstein, J., B. Everett, and L. Feng. 1996. \textit{Navigating Mobile Robots: Systems and
Techniques. Wellesley,} MA: A.K. Peters, Ltd.

Borenstein, J., and Y. Koren. 1991. The vector field histogram – fast obstacle avoidance
for mobile robots. \textit{IEEE Transactions on Robotics and Automation} 7:278–288.

Bosse, M., P. Newman, J. Leonard, M. Soika, W. Feiten, and S. Teller. 2004. Simultaneous
localization and map building in large-scale cyclic environments using the
atlas framework. \textit{International Journal of Robotics Research} 23:1113–1139.

Bosse, M., P. Newman, M. Soika, W. Feiten, J. Leonard, and S. Teller. 2003. An atlas
framework for scalable mapping. In \textit{Proceedings of the International Conference on
Robotics and Automation (ICRA)}.

Bouguet, J.-Y., and P. Perona. 1995. Visual navigation using a single camera. In
\textit{Proceedings of the International Conference on Computer Vision (ICCV)}, pp. 645–652.

Boutilier, C., R. Brafman, and C. Geib. 1998. Structured reachability analysis for
Markov decision processes. In \textit{Proceedings of the Conference on Uncertainty in AI
(UAI)}, pp. 24–32.

Brafman, R.I. 1997. A heuristic variable grid solution method for POMDPs. In \textit{Proceedings
of the AAAI National Conference on Artificial Intelligence.}

Brooks, R.A. 1986. A robust layered control system for a mobile robot. \textit{IEEE Journal
of Robotics and Automation} 2:14–23.

Brooks, R.A. 1990. Elephants don’t play chess. \textit{Autonomous Robots} 6:3–15.

Brooks, R.A., and T. Lozano-Perez. 1985. A subdivision algorithm in configuration
space for findpath with rotation. \textit{IEEE Transactions on Systems, Man, and Cybernetics}
15:224–233.

Bryson, A.E., and H. Yu-Chi. 1975. \textit{Applied Optimal Control.} Halsted Press, JohnWiley
\& Sons.

Bulata, H., and M. Devy. 1996. Incremental construction of a landmark-based and
topological model of indoor environments by a mobile robot. In \textit{Proceedings of the
International Conference on Robotics and Automation (ICRA)}, Minneapolis, USA.

Burgard, W., A.B. Cremers, D. Fox, D. Hähnel, G. Lakemeyer, D. Schulz, W. Steiner,
and S. Thrun. 1999a. Experiences with an interactive museum tour-guide robot.
\textit{Artificial Intelligence} 114:3–55.

Burgard,W., A. Derr, D. Fox, and A.B. Cremers. 1998. Integrating global position estimation
and position tracking for mobile robots: the Dynamic Markov Localization
approach. In \textit{Proceedings of the IEEE/RSJ Int. Conf. on Intelligent Robots and Systems
(IROS)}.

Burgard, W., D. Fox, D. Hennig, and T. Schmidt. 1996. Estimating the absolute position
of a mobile robot using position probability grids. In \textit{Proceedings of the National
Conference on Artificial Intelligence (AAAI)}.

Burgard,W., D. Fox, H. Jans, C. Matenar, and S. Thrun. 1999b. Sonar-based mapping
of large-scale mobile robot environments using EM. In \textit{Proceedings of the International
Conference on Machine Learning,} Bled, Slovenia.

Burgard, W., D. Fox, M. Moors, R.G. Simmons, and S. Thrun. 2000. Collaborative
multi-robot exploration. In \textit{Proceedings of the International Conference on Robotics and
Automation (ICRA)}.

Burgard, W., D. Fox, and S. Thrun. 1997. Active mobile robot localization. In \textit{Proceedings
of the Fourteenth International Joint Conference on Artificial Intelligence (IJCAI)},
San Mateo, CA. Morgan Kaufmann.

Burgard,W.,M. Moors, C. Stachniss, and F. Schneider. 2004. Coordinated multi-robot
exploration. \textit{IEEE Transactions on Robotics and Automation}. To appear.

Canny, J. 1987. \textit{The Complexity of Robot Motion Planning.} Cambridge, MA: MIT Press.

Casella, G.C., and R.L. Berger. 1990. \textit{Statistical Inference.} Pacific Grove, CA:
Wadsworth \& Brooks.

Cassandra, A.R., L.P. Kaelbling, and M.L. Littman. 1994. Acting optimally in partially
observable stochastic domains. In \textit{Proceedings of the AAAI National Conference on
Artificial Intelligence}, pp. 1023–1028.

Cassandra, A., M. Littman, and N. Zhang. 1997. Incremental pruning: A simple, fast,
exact method for partially observable Markov decision processes. In \textit{Proceedings of
the Conference on Uncertainty in AI (UAI)}.

Castellanos, J.A., J.M.M. Montiel, J. Neira, and J.D. Tardós. 1999. The SPmap: A
probabilistic framework for simultaneous localization and map building. \textit{IEEE
Transactions on Robotics and Automation} 15:948–953.

Castellanos, J.A., J. Neira, and J.D. Tardós. 2001. Multisensor fusion for simultaneous
localization and map building. \textit{IEEE Transactions on Robotics and Automation} 17:
908–914.

Castellanos, J.A., J. Neira, and J.D. Tardós. 2004. Limits to the consistency of the EKFbased
SLAM. In M.I. Ribeiro and J. Santos-Victor (eds.), \textit{Proceedings of Intelligent
Autonomous Vehicles (IAV-2004)}, Lisboa, PT. IFAC/EURON and IFAC/Elsevier.

Chatila, R., and J.-P. Laumond. 1985. Position referencing and consistent world modeling
for mobile robots. In \textit{Proceedings of the International Conference on Robotics and
Automation (ICRA)}, pp. 138–145.

Cheeseman, P., and P. Smith. 1986. On the representation and estimation of spatial
uncertainty. \textit{International Journal of Robotics} 5:56 – 68.

Choset, H. 1996. \textit{Sensor Based Motion Planning: The Hierarchical Generalized Voronoi
Graph}. PhD thesis, California Institute of Technology.

Choset, H. 2001. Coverage for robotics—a survey of recent results. \textit{Annals of Mathematical
Artificial Intelligence} 31:113–126.

Choset, H., K. Lynch, S. Hutchinson, G. Kantor,W. Burgard, L. Kavraki, and S. Thrun.
2004. \textit{Principles of Robotic Motion: Theory, Algorithms, and Implementation.} Cambridge,
MA: MIT Press.

Chown, E., S. Kaplan, and D. Kortenkamp. 1995. Prototypes, location, and associative
networks (plan): Towards a unified theory of cognitive mapping. \textit{Cognitive Science}
19:1–51.

Chrisman, L. 1992. Reinforcement learning with perceptual aliasing: The perceptual
distinction approach. In \textit{Proceedings of 1992 AAAI Conference}, Menlo Park, CA.
AAAI Press / The MIT Press.

Cid, R.M., C. Parra, and M. Devy. 2002. Visual navigation in natural environments:
from range and color data to a landmark-based model.\textit{ Autonomous Robots} 13:143–
168.

Cohn, D. 1994. Queries and exploration using optimal experiment design. In J.D.
Cowan, G. Tesauro, and J. Alspector (eds.), \textit{Advances in Neural Information Processing
Systems 6}, San Mateo, CA. Morgan Kaufmann.

Connell, J. 1990. \textit{Minimalist Mobile Robotics.} Boston: Academic Press.

Coppersmith, D., and S. Winograd. 1990. Matrix multiplication via arithmetic progressions.
\textit{Journal of Symbolic Computation} 9:251–280.

Cover, T.M., and J.A. Thomas. 1991. \textit{Elements of Information Theory}. Wiley.

Cowell, R.G., A.P. Dawid, S.L. Lauritzen, and D.J. Spiegelhalter. 1999. \textit{Probabilistic
Networks and Expert Systems}. Berlin, New York: Springer Verlag.

Cox, I.J. 1991. Blanche—an experiment in guidance and navigation of an autonomous
robot vehicle. \textit{IEEE Transactions on Robotics and Automation} 7:193–204.

Cox, I.J., and J.J. Leonard. 1994. Modeling a dynamic environment using a Bayesian
multiple hypothesis approach. \textit{Artificial Intelligence} 66:311–344.

Cox, I.J., and G.T. Wilfong (eds.). 1990. \textit{Autonomous Robot Vehicles.} Springer Verlag.

Craig, J.J. 1989. \textit{Introduction to Robotics: Mechanics and Control (2nd Edition)}. Reading,
MA: Addison-Wesley Publishing, Inc. 3rd edition.

Crowley, J. 1989. World modeling and position estimation for a mobile robot using
ultrasonic ranging. In \textit{Proceedings of the International Conference on Robotics and
Automation (ICRA)}, pp. 674–680.

Csorba, M. 1997. \textit{Simultaneous Localisation and Map Building.} PhD thesis, University
of Oxford.

Davison, A. 1998. \textit{Mobile Robot Navigation Using Active Vision.} PhD thesis, University
of Oxford, Oxford, UK.

Davison, A. 2003. Real time simultaneous localisation and mapping with a single
camera. In \textit{Proceedings of the International Conference on Computer Vision (ICCV)},
Nice, France.

Dean, L.P. Kaelbling, J. Kirman, and A. Nicholson. 1995. Planning under time constraints
in stochastic domains. \textit{Artificial Intelligence} 76:35–74.

Deans, M., and M. Hebert. 2000. Invariant filtering for simultaneous localization and
mapping. In \textit{Proceedings of the International Conference on Robotics and Automation
(ICRA)}, pp. 1042–1047.

Deans, M.C., and M. Hebert. 2002. Experimental comparison of techniques for localization
and mapping using a bearing-only sensor. In \textit{Proceedings of the International
Symposium on Experimental Robotics (ISER)}, Sant’Angelo d’Ischia, Italy.

Dearden, R., and C. Boutilier. 1994. Integrating planning and execution in stochastic
domains. In \textit{Proceedings of the AAAI Spring Symposium on Decision Theoretic Planning,}
pp. 55–61, Stanford, CA.

Dedeoglu, G., and G. Sukhatme. 2000. Landmark-based matching algorithm for
cooperative mapping by autonomous robots. In \textit{Proceedings of the International
Symposium on Distributed Autonomous Robotic Systems (DARS 2000)}, Knoxville, Tenneessee.

DeGroot, Morris H. 1975. \textit{Probability and Statistics.} Reading, MA: Addison-Wesley.

Dellaert, F. 2005. Square root SAM. In S. Thrun, G. Sukhatme, and S. Schaal (eds.), \textit{Proceedings
of the Robotics Science and Systems Conference.} Cambridge, MA: MIT Press.

Dellaert, F., D. Fox,W. Burgard, and S. Thrun. 1999. Monte Carlo localization for mobile
robots. In \textit{Proceedings of the International Conference on Robotics and Automation
(ICRA)}.

Dellaert, F., S.M. Seitz, C. Thorpe, and S. Thrun. 2003. EM,MCMC, and chain flipping
for structure from motion with unknown correspondence. \textit{Machine Learning} 50:45–
71.

Dempster, A.P., A.N. Laird, and D.B. Rubin. 1977. Maximum likelihood from incomplete
data via the EM algorithm. \textit{Journal of the Royal Statistical Society, Series B} 39:
1–38.

Deng, X., and C. Papadimitriou. 1998. How to learn in an unknown environment:
The rectilinear case. \textit{Journal of the ACM} 45:215–245.

Devroye, L., L. Györfi, and G. Lugosi. 1996. \textit{A Probabilistic Theory of Pattern Recognition.}
New York, NY: Springer-Verlag.

Devy, M., and H. Bulata. 1996. Multi-sensory perception and heterogeneous representations
for the navigation of a mobile robot in a structured environment. In
\textit{Proceedings of the Symposium on Intelligent Robot Systems}, Lisboa.

Devy, M., and C. Parra. 1998. 3-d scene modelling and curve-based localization in
natural environments. In \textit{Proceedings of the International Conference on Robotics and
Automation (ICRA)}.

Dias, M.B., M. Zinck, R. Zlot, and A. Stentz. 2004. Robust multirobot coordination in
dynamic environments. In \textit{Proceedings of the International Conference on Robotics and
Automation (ICRA)}.

Dickmanns, E.D. 2002. Vision for ground vehicles: history and prospects. \textit{International
Journal of Vehicle Autonomous Systems }1:1–44.

Dickmanns, E.D., and V. Graefe. 1988. Application of monocular machine vision.
\textit{Machine Vision and Applications} 1:241–261.

Diebel, J., K. Reuterswärd, J. Davis, and S. Thrun. 2004. Simultaneous localization
and mapping with active stereo vision. In \textit{Proceedings of the IEEE/RSJ Int. Conf. on
Intelligent Robots and Systems (IROS)}.

Dietterich, T.G. 2000. Hierarchical reinforcement learning with the MAXQ value
function decomposition. \textit{Journal of Artificial Intelligence Research} 13:227–303.

Dissanayake, G., P. Newman, S. Clark, H.F. Durrant-Whyte, and M. Csorba. 2001. A
solution to the simultaneous localisation and map building (SLAM) problem. \textit{IEEE
Transactions on Robotics and Automation} 17:229–241.

Dissanayake, G., S.B. Williams, H. Durrant-Whyte, and T. Bailey. 2002. Map management
for efficient simultaneous localization and mapping (SLAM). \textit{Autonomous
Robots} 12:267–286.

Dorf, R.C., and R.H. Bishop. 2001. \textit{Modern Control Systems (Ninth Edition)}. Englewood
Cliffs, NJ: Prentice Hall.

Doucet, A. 1998. On sequential simulation-based methods for Bayesian filtering.
Technical Report CUED/F-INFENG/TR 310, Cambridge University, Department
of Engineering, Cambridge, UK.

Doucet, A., J.F.G. de Freitas, and N.J. Gordon (eds.). 2001. \textit{Sequential Monte Carlo
Methods In Practice.} New York: Springer Verlag.

Driankov, D., and A. Saffiotti (eds.). 2001. \textit{Fuzzy Logic Techniques for Autonomous
Vehicle Navigation}, volume 61 of \textit{Studies in Fuzziness and Soft Computing}. Berlin,
Germany: Springer-Verlag.

Duckett, T., S. Marsland, and J. Shapiro. 2000. Learning globally consistent maps by
relaxation. In \textit{Proceedings of the International Conference on Robotics and Automation
(ICRA)}, pp. 3841–3846.

Duckett, T., S. Marsland, and J. Shapiro. 2002. Fast, on-line learning of globally
consistent maps. \textit{Autonomous Robots} 12:287 – 300.

Duckett, T., and U. Nehmzow. 2001. Mobile robot self-localisation using occupancy
histograms and a mixture of Gaussian location hypotheses. \textit{Robotics and
Autonomous Systems }34:119–130.

Duda, R.O., P.E. Hart, and D. Stork. 2000. \textit{Pattern classification and scene analysis (2nd
edition)}. New York: JohnWiley and Sons.

Dudek, G., and D. Jegessur. 2000. Robust place recognition using local appearance
based methods. In \textit{Proceedings of the International Conference on Robotics and Automation
(ICRA)}, pp. 466–474.

Dudek, G., and M. Jenkin. 2000. \textit{Computational Principles of Mobile Robotics.} Cambridge
CB2 2RU, UK: Cambridge University Press.

Dudek, G., M. Jenkin, E. Milios, and D. Wilkes. 1991. Robotic exploration as graph
construction. \textit{IEEE Transactions on Robotics and Automation} 7:859–865.

Durrant-Whyte, H.F. 1988. Uncertain geometry in robotics. \textit{IEEE Transactions on
Robotics and Automation} 4:23 – 31.

Durrant-Whyte, H.F. 1996. Autonomous guided vehicle for cargo handling applications.
\textit{International Journal of Robotics Research} 15.

Durrant-Whyte, H., S. Majumder, S. Thrun, M. de Battista, and S. Scheding. 2001. A
Bayesian algorithm for simultaneous localization and map building. In \textit{Proceedings
of the 10th International Symposium of Robotics Research (ISRR’01)}, Lorne, Australia.

Elfes, A. 1987. Sonar-based real-world mapping and navigation. \textit{IEEE Transactions on
Robotics and Automation} pp. 249–265.

Eliazar, A., and R. Parr. 2003. DP-SLAM: Fast, robust simultaneous localization and
mapping without predetermined landmarks. In \textit{Proceedings of the Sixteenth International
Joint Conference on Artificial Intelligence (IJCAI)}, Acapulco, Mexico. IJCAI.

Eliazar, A., and R. Parr. 2004. DP-SLAM 2.0. In \textit{Proceedings of the International Conference
on Robotics and Automation (ICRA)}, New Orleans, USA.

Elliott, R.J., L.Aggoun, and J.B. Moore. 1995. \textit{Hidden Markov Models: Estimation and
Control.} New York, NY: Springer-Verlag.

Engelson, S., and D. McDermott. 1992. Error correction in mobile robot map learning.
In \textit{Proceedings of the International Conference on Robotics and Automation (ICRA)}, pp.
2555–2560.

Etter, P.C. 1996. \textit{Underwater Acoustic Modeling: Principles, Techniques and Applications.}
Amsterdam: Elsevier.

Featherstone, R. 1987.\textit{ Robot Dynamics Algorithms.} Boston, MA: Kluwer Academic
Publishers.

Feder, H.J.S., J.J. Leonard, and C.M. Smith. 1999. Adaptive mobile robot navigation
and mapping. \textit{International Journal of Robotics Research} 18:650–668.

Feldman, D. 1962. Contributions to the two-armed bandit problem. \textit{Ann. Math. Statist}
33:847–856.

Feller,W. 1968. \textit{An Introduction To Probability Theory And Its Applications (3rd edition)x.}
Quinn-Woodbine.

Feng, L., J. Borenstein, and H.R. Everett. 1994. “Where am I?” Sensors and methods
for autonomous mobile robot positioning. Technical Report UM-MEAM-94-12,
University of Michigan, Ann Arbor, MI.

Fenwick, J., P. Newman, and J. Leonard. 2002. Collaborative concurrent mapping and
localization. In \textit{Proceedings of the International Conference on Robotics and Automation
(ICRA)}.

Ferguson, D., T. Stentz, and S. Thrun. 2004. PAO* for planning with hidden state. In
\textit{Proceedings of the International Conference on Robotics and Automation (ICRA).}

Fischler, M.A., and R.C. Bolles. 1981. Random sample consensus: A paradigm
for model fitting with applications to image analysis and automated cartography.
\textit{Communications of the ACM} 24:381–395.

Folkesson, J., and H.I. Christensen. 2003. Outdoor exploration and SLAM using a
compressed filter. In \textit{Proceedings of the IEEE International Conference on Robotics and
Automation (ICRA)}, pp. 419–427.

Folkesson, J., and H.I. Christensen. 2004a. Graphical SLAM: A self-correcting map.
In \textit{Proceedings of the International Conference on Robotics and Automation (ICRA)}.

Folkesson, J., and H.I. Christensen. 2004b. Robust SLAM. In \textit{Proceedings of the International
Symposium on Autonomous Vehicles}, Lisboa, PT.

Fox, D. 2003. Adapting the sample size in particle filters through KLD-sampling.
\textit{International Journal of Robotics Research} 22:985 – 1003.

Fox, D., W. Burgard, F. Dellaert, and S. Thrun. 1999a. Monte Carlo localization: Efficient
position estimation for mobile robots. In \textit{Proceedings of the National Conference
on Artificial Intelligence (AAAI)}, Orlando, FL. AAAI.

Fox, D., W. Burgard, H. Kruppa, and S. Thrun. 2000. A probabilistic approach to
collaborative multi-robot localization. \textit{Autonomous Robots} 8.

Fox, D., W. Burgard, and S. Thrun. 1998. Active Markov localization for mobile
robots. \textit{Robotics and Autonomous Systems} 25:195–207.

Fox, D., W. Burgard, and S. Thrun. 1999b. Markov localization for mobile robots in
dynamic environments. \textit{Journal of Artificial Intelligence Research (JAIR)} 11:391–427.

Fox, D., W. Burgard, and S. Thrun. 1999c. Markov localization for mobile robots in
dynamic environments. \textit{Journal of Artificial Intelligence Research} 11:391–427.

Fox, D., J. Ko, K. Konolige, and B. Stewart. 2005. A hierarchical Bayesian approach to
mobile robot map structure learning. In P. Dario and R. Chatila (eds.), \textit{Robotics Research:
The Eleventh International Symposium, Springer Tracts in Advanced Robotics
(STAR)}. Springer Verlag.

Freedman, D., and P. Diaconis. 1981. On this histogram as a density estimator: L\_2
theory. \textit{Zeitschrift für Wahrscheinlichkeitstheorie und verwandte Gebiete} 57:453–476.

Frese, U. 2004. \textit{An O(logn) Algorithm for Simultaneous Localization and Mapping of
Mobile Robots in Indoor Environments.} PhD thesis, University of Erlangen-Nürnberg,
Germany.

Frese, U., and G. Hirzinger. 2001. Simultaneous localization and mapping—a discussion.
In \textit{Proceedings of the IJCAI Workshop on Reasoning with Uncertainty in Robotics},
pp. 17–26, Seattle,WA.

Frese, U., P. Larsson, and T. Duckett. 2005. A multigrid algorithm for simultaneous
localization and mapping. \textit{IEEE Transactions on Robotics.} To appear.

Frueh, C., and A. Zakhor. 2003. Constructing 3d city models by merging groundbased
and airborne views. In \textit{Proceedings of the IEEE Computer Society Conference on
Computer Vision and Pattern Recognition (CVPR)}, Madison, Wisconsin.

Gat, E. 1998. Three-layered architectures. In D. Kortenkamp, R.P. Bonasso, and
R. Murphy (eds.), \textit{AI-based Mobile Robots: Case Studies of Successful Robot Systems},
pp. 195–210. Cambridge, MA: MIT Press.

Gat, E., R. Desai, R. Ivlev, J. Loch, and D.P. Miller. 1994. Behavior control for robotic
exploration of planetary surfaces. \textit{IEEE Transactions on Robotics and Automation} 10:
490–503.

Gauss, K.F. 1809. \textit{Theoria Motus Corporum Coelestium (Theory of the Motion of the Heavenly
Bodies Moving about the Sun in Conic Sections).} Republished in 1857, and by
Dover in 1963: Little, Brown, and Co.

Geffner, H., and B. Bonet. 1998. Solving large POMDPs by real time dynamic programming.
In\textit{ Working Notes Fall AAAI Symposium on POMDPs}, Stanford, CA.

Gerkey, B., S. Thrun, and G. Gordon. 2004. Parallel stochastic hill-climbing with
small teams. In L. Parker, F. Schneider, and A. Schultz (eds.), \textit{Proceedings of the 3rd
InternationalWorkshop on Multi-Robot Systems}, Amsterdam. NRL, Kluwer Publisher.

Gilks, W.R., S. Richardson, and D.J. Spiegelhalter (eds.). 1996. \textit{Markov Chain Monte
Carlo in Practice.} Chapman and Hall/CRC.

Goldberg, K. 1993. Orienting polygonal parts without sensors. \textit{Algorithmica} 10:201–
225.

Golfarelli, M., D. Maio, and S. Rizzi. 1998. Elastic correction of dead-reckoning errors
in map building. In \textit{Proceedings of the IEEE/RSJ Int. Conf. on Intelligent Robots and
Systems (IROS)}, pp. 905–911.

Golub, G.H., and C.F. Van Loan. 1986. \textit{Matrix Computations}. North Oxford Academic.

González-Baños, H.H., and J.C. Latombe. 2001. Navigation strategies for exploring
indoor environments. \textit{International Journal of Robotics Research.}

Gordon, G. J. 1995. Stable function approximation in dynamic programming. In
A. Prieditis and S. Russell (eds.), \textit{Proceedings of the Twelfth International Conference
on Machine Learning.} Also appeared as Technical Report CMU-CS-95-103, Carnegie
Mellon University, School of Computer Science, Pittsburgh, PA.

Greiner, R., and R. Isukapalli. 1994. Learning to select useful landmarks. In \textit{Proceedings
of 1994 AAAI Conference}, pp. 1251–1256, Menlo Park, CA. AAAI Press / The
MIT Press.

Grunbaum, F.A., M.Bernfeld, and R.E. Blahut (eds.). 1992. \textit{Radar and Sonar: Part II}.
New York, NY: Springer-Verlag.

Guibas, L.J., D.E. Knuth, and M. Sharir. 1992. Randomized incremental construction
of Delaunay and Voronoi diagrams. \textit{Algorithmica} 7:381–413. \textit{See also 17th Int. Coll.
on Automata, Languages and Programming,} 1990, pp. 414–431.

Guibas, L.J., J.-C. Latombe, S.M. LaValle, D. Lin, and R. Motwani. 1999. A visibilitybased
pursuit-evasion problem. \textit{International Journal of Computational Geometry and
Applications} 9:471–493.

Guivant, J., and E. Nebot. 2001. Optimization of the simultaneous localization and
map building algorithm for real time implementation. \textit{IEEE Transactions on Robotics
and Automation} 17:242–257. In press.

Guivant, J., and E. Nebot. 2002. Improving computational and memory requirements
of simultaneous localization and map building algorithms. In \textit{Proceedings of the
International Conference on Robotics and Automation (ICRA)}, pp. 2731–2736.

Guivant, J., E. Nebot, and S. Baiker. 2000. Autonomous navigation and map building
using laser range sensors in outdoor applications. \textit{Journal of Robotics Systems} 17:
565–583.

Guivant, J.E., E.M. Nebot, J. Nieto, and F. Masson. 2004. Navigation and mapping in
large unstructured environments. \textit{International Journal of Robotics Research} 23.

Gutmann, J.S., and D. Fox. 2002. An experimental comparison of localization methods
continued. In \textit{Proc. of the IEEE/RSJ International Conference on Intelligent Robots
and Systems (IROS)}.

Gutmann, J.-S.,W. Burgard, D. Fox, and K. Konolige. 1998. An experimental comparison
of localization methods. In \textit{Proceedings of the IEEE/RSJ Int. Conf. on Intelligent
Robots and Systems (IROS)}.

Gutmann, J.-S., and K. Konolige. 2000. Incremental mapping of large cyclic environments.
In \textit{Proceedings of the IEEE International Symposium on Computational Intelligence
in Robotics and Automation (CIRA)}.

Gutmann, J.-S., and B. Nebel. 1997. Navigation mobiler roboter mit laserscans. In
\textit{Autonome Mobile Systeme}. Berlin: Springer Verlag. In German.

Gutmann, J.-S., and C. Schlegel. 1996. AMOS: Comparison of scan matching approaches
for self-localization in indoor environments. In \textit{Proc. of the 1st Euromicro
Workshop on Advanced Mobile Robots}. IEEE Computer Society Press.

Hähnel, D.,W. Burgard, B.Wegbreit, and S. Thrun. 2003a. Towards lazy data association
in SLAM. In \textit{Proceedings of the 11th International Symposium of Robotics Research
(ISRR’03)}, Sienna, Italy. Springer.

Hähnel, D., D. Fox, W. Burgard, and S. Thrun. 2003b. A highly efficient FastSLAM
algorithm for generating cyclic maps of large-scale environments from raw laser
range measurements. In \textit{Proceedings of the IEEE/RSJ Int. Conf. on Intelligent Robots
and Systems (IROS).}

Hähnel, D., D. Schulz, and W. Burgard. 2003c. Mobile robot mapping in populated
environments. \textit{Autonomous Robots} 17:579–598.

Hartley, R., and A. Zisserman. 2000. \textit{Multiple View Geometry in Computer Vision.} Cambridge
University Press.

Hauskrecht, M. 1997. Incremental methods for computing bounds in partially observable
Markov decision processes. In \textit{Proceedings of the AAAI National Conference
on Artificial Intelligence}, pp. 734–739, Providence, RI.

Hauskrecht, M. 2000. Value-function approximations for partially observable Markov
decision processes. \textit{Journal of Artificial Intelligence Research }13:33–94.

Hayet, J.B., F. Lerasle, and M. Devy. 2002. A visual landmark framework for indoor
mobile robot navigation. In \textit{Proceedings of the International Conference on Robotics and
Automation (ICRA)},Washington, DC.

Hertzberg, J., and F. Kirchner. 1996. Landmark-based autonomous navigation in
sewerage pipes. In \textit{Proc. of the First Euromicro Workshop on Advanced Mobile Robots.}

Hinkel, R., and T. Knieriemen. 1988. Environment perception with a laser radar in
a fast moving robot. In \textit{Proceedings of Symposium on Robot Control,} pp. 68.1–68.7,
Karlsruhe, Germany.

Hoey, J., R. St-Aubin, A. Hu, and C. Boutilier. 1999. SPUDD: Stochastic planning
using decision diagrams. In \textit{Proceedings of the Conference on Uncertainty in AI (UAI)},
pp. 279–288.

Höllerer, T., S. Feiner, T. Terauchi, G. Rashid, and D. Hallaway. 1999. Exploring
MARS: Developing indoor and outdoor user interfaces to a mobile augmented reality
system. \textit{Computers and Graphics} 23:779–785.

Howard, A. 2004. Multi-robot mapping using manifold representations. In \textit{Proceedings
of the International Conference on Robotics and Automation (ICRA)}, pp. 4198–4203.

Howard, A., M.J. Mataric, and G.S. Sukhatme. 2002. An incremental deployment
algorithm for mobile robot teams. In \textit{Proceedings of the IEEE/RSJ Int. Conf. on Intelligent
Robots and Systems (IROS).}

Howard, A., M.J. Mataric, and G.S. Sukhatme. 2003. Cooperative relative localization
for mobile robot teams: An ego-centric approach. In \textit{Proceedings of the Naval
Research Laboratory Workshop on Multi-Robot Systems,}Washington, D.C.

Howard, A., L.E. Parker, and G.S. Sukhatme. 2004. The SDR experience: Experiments
with a large-scale heterogenous mobile robot team. In \textit{Proceedings of the 9th
International Symposium on Experimental Robotics 2004}, Singapore.

Howard, R.A. 1960. \textit{Dynamic Programming and Markov Processes.} MIT Press andWiley.

Iagnemma, K., and S. Dubowsky. 2004. \textit{Mobile Robots in Rough Terrain: Estimation,
Motion Planning, and Control with Application to Planetary Rovers.} Springer.

Ilon, B.E., 1975. Wheels for a course stable selfpropelling vehicle movable in any desired
direction on the ground or some other base. United States Patent \#3,876,255.

Iocchi, L., K. Konolige, and M. Bajracharya. 2000. Visually realistic mapping of a
planar environment with stereo. In \textit{Proceesings of the 2000 International Symposium
on Experimental Robotics,} Waikiki, Hawaii.

IRobots Inc., 2004. Roomba robotic floor vac. On the Web at
http://www.irobot.com/consumer/.

Isaacs, R. 1965. \textit{Differential Games–A Mathematical Theory with Applications to Warfare
and Pursuit, Control and Optimization.} John Wiley and Sons, Inc.

Isard, M., and A. Blake. 1998. CONDENSATION: conditional density propagation
for visual tracking. \textit{International Journal of Computer Vision} 29:5–28.

Jaeger, H. 2000. Observable operator processes and conditioned continuation representations.
\textit{Neural Computation} 12:1371–1398.

James, M., and S. Singh. 2004. Learning and discovery of predictive state representations
in dynamical systems with reset. In \textit{Proceedings of the Twenty-First International
Conference on Machine Learning (ICML)}, pp. 417–424.

Jazwinsky, A.M. 1970. \textit{Stochastic Processes and Filtering Theory}. New York: Academic.

Jensfelt, P., D. Austin, O. Wijk, and M. Andersson. 2000. Feature based condensation
for mobile robot localization. In \textit{Proceedings of the International Conference on Robotics
and Automation (ICRA)}, pp. 2531–2537.

Jensfelt, P., and H.I. Christensen. 2001a. Active global localisation for a mobile robot
using multiple hypothesis tracking. \textit{IEEE Transactions on Robotics and Automation}
17:748–760.

Jensfelt, P., and H.I. Christensen. 2001b. Pose tracking using laser scanning and
minimalistic environmental models. \textit{IEEE Transactions on Robotics and Automation}
17:138–147.

Jensfelt, P., H.I. Christensen, and G. Zunino. 2002. Integrated systems for mapping
and localization. In \textit{Proceedings of the International Conference on Robotics and Automation
(ICRA).}

Julier, S., and J. Uhlmann. 1997. A new extension of the Kalman filter to nonlinear
systems. In \textit{International Symposium on Aerospace/Defense Sensing, Simulate and
Controls,} Orlando, FL.

Julier, S.J., and J.K. Uhlmann. 2000. Building a million beacon map. In \textit{Proceedings of
the SPIE Sensor Fusion and Decentralized Control in Robotic Systems IV, Vol. \#4571.}

Jung, I.K., and S. Lacroix. 2003. High resolution terrain mapping using low altitude
aerial stereo imagery. In \textit{Proceedings of the International Conference on Computer Vision
(ICCV)}, Nice, France.

Kaelbling, L.P., A.R. Cassandra, and J.A. Kurien. 1996. Acting under uncertainty: Discrete
Bayesian models for mobile-robot navigation. In \textit{Proceedings of the IEEE/RSJ
Int. Conf. on Intelligent Robots and Systems (IROS)}.

Kaelbling, L.P., M.L. Littman, and A.R. Cassandra. 1998. Planning and acting in
partially observable stochastic domains. \textit{Artificial Intelligence} 101:99–134.

Kaelbling, L. P., and S. J. Rosenschein. 1991. Action and planning in embedded agents.
In \textit{Designing Autonomous Agents}, pp. 35–48. Cambridge, MA: The MIT Press (and
Elsevier).

Kalman, R.E. 1960. A new approach to linear filtering and prediction problems.
\textit{Trans. ASME, Journal of Basic Engineering} 82:35–45.

Kanazawa, K., D. Koller, and S.J. Russell. 1995. Stochastic simulation algorithms
for dynamic probabilistic networks. In \textit{Proceedings of the 11th Annual Conference on
Uncertainty in AI,} Montreal, Canada.

Kavraki, L., and J.-C. Latombe. 1994. Randomized preprocessing of configuration
space for fast path planning. In \textit{Proceedings of the International Conference on Robotics
and Automation (ICRA)}, pp. 2138–2145.

Kavraki, L., P. Svestka, J.-C. Latombe, and M. Overmars. 1996. Probabilistic roadmaps
for path planning in high-dimensional configuration spaces. \textit{IEEE Transactions on
Robotics and Automation} 12:566–580.

Kearns, M., and S. Singh. 2003. Near-optimal reinforcement learning in polynomial
time. \textit{Machine Learning} 49:209–232.

Khatib, O. 1986. Real-time obstacle avoidance for robot manipulator and mobile
robots. \textit{The International Journal of Robotics Research} 5:90–98.

Kirk, R.E., and P. Kirk. 1995. \textit{Experimental Design: Procedures for the Behavioral Sciences.}
Pacific Grove, CA: Brooks/Cole.

Kitagawa, G. 1996. Monte Carlo filter and smoother for non-Gaussian nonlinear state
space models. \textit{Journal of Computational and Graphical Statistics }5:1–25.

Kleinberg, J. 1994. The localization problem for mobile robots. In \textit{Proc. of the 35th
IEEE Symposium on Foundations of Computer Science.}

Ko, J., B. Stewart, D. Fox, K. Konolige, and B. Limketkai. 2003. A practical, decisiontheoretic
approach to multi-robot mapping and exploration. In \textit{Proc. of the IEEE/RSJ
International Conference on Intelligent Robots and Systems (IROS)}, pp. 3232–3238.

Koditschek, D.E. 1987. Exact robot navigation by means of potential functions: Some
topological considerations. In \textit{Proceedings of the International Conference on Robotics
and Automation (ICRA)}, pp. 1–6.

Koenig, S., and R.G. Simmons. 1993. Exploration with and without a map. In \textit{Proceedings
of the AAAIWorkshop on Learning Action Models at the Eleventh National Conference
on Artificial Intelligence (AAAI)}, pp. 28–32. Also available as AAAI Technical
Report WS-93-06.

Koenig, S., and R. Simmons. 1998. A robot navigation architecture based on partially
observable Markov decision process models. In Kortenkamp et al. (1998).

Koenig, S., and C. Tovey. 2003. Improved analysis of greedy mapping. In \textit{Proceedings
of the IEEE/RSJ Int. Conf. on Intelligent Robots and Systems (IROS)}.

Konecny, G. 2002. \textit{Geoinformation: Remote Sensing, Photogrammetry and Geographical
Information Systems}. Taylor \& Francis.

Konolige, K. 2004. Large-scale map-making. In \textit{Proceedings of the AAAI National
Conference on Artificial Intelligence,} pp. 457–463, San Jose, CA. AAAI.

Konolige, K., and K. Chou. 1999. Markov localization using correlation. In \textit{Proceedings
of the International Joint Conference on Artificial Intelligence (IJCAI)}.

Konolige, K., D Fox, C. Ortiz, A. Agno, M. Eriksen, B. Limketkai, J. Ko, B. Morisset,
D. Schulz, B. Stewart, and R. Vincent. 2005. Centibots: Very large scale distributed
robotic teams. In M. Ang and O. Khatib (eds.), \textit{Experimental Robotics: The 9th International
Symposium}, Springer Tracts in Advanced Robotics (STAR). Springer Verlag.

Konolige, K., J.-S. Gutmann, D. Guzzoni, R. Ficklin, and K. Nicewarner. 1999. A
mobile robot sense net. In \textit{Proceedings of SPIE 3839 Sensor Fusion and Decentralized
Control in Robotic Systmes II,} Boston.

Korf, R.E. 1988. Real-time heuristic search: New results. In \textit{Proceedings of the sixth
National Conference on Artificial Intelligence (AAAI-88)}, pp. 139–143, Los Angeles, CA
90024. Computer Science Department, University of California, AAAI Press/MIT
Press.

Kortenkamp, D., R.P. Bonasso, and R. Murphy (eds.). 1998. \textit{Artificial Intelligence and
Mobile Robots: Case Studies of Successful Robot Systems.} Cambridge, MA:MIT/AAAI
Press.

Kortenkamp, D., and T. Weymouth. 1994. Topological mapping for mobile robots
using a combination of sonar and vision sensing. In \textit{Proceedings of the Twelfth National
Conference on Artificial Intelligence,} pp. 979–984, Menlo Park. AAAI, AAAI
Press/MIT Press.

Kröse, B., N. Vlassis, and R. Bunschoten. 2002. Omnidirectional vision for
appearance-based robot localization. In G.D. Hagar, H.I. Cristensen, H. Bunke,
and R. Klein (eds.), \textit{Sensor Based Intelligent Robots (Lecture Notes in Computer Science
\#2238)}, pp. 39–50. Springer Verlag.

Krotkov, E., M. Hebert, L. Henriksen, P. Levin, M. Maimone, R.G. Simmons, and
J. Teza. 1999. Evolution of a prototype lunar rover: Addition of laser-based hazard
detection, and results from field trials in lunar analog terrain. \textit{Autonomous Robots}
7:119–130.

Kuipers, B., and Y.-T. Byun. 1990. A robot exploration and mapping strategy based
on a semantic hierarchy of spatial representations. Technical report, Department
of Computer Science, University of Texas at Austin, Austin, Texas 78712.

Kuipers, B., and Y.-T. Byun. 1991. A robot exploration and mapping strategy based
on a semantic hierarchy of spatial representations. \textit{Robotics and Autonomous Systems}
pp. 47–63.

Kuipers, B.J., and T.S. Levitt. 1988. Navigation and mapping in large-scale space.\textit{ AI
Magazine.}

Kuipers, B., J. Modayil, P. Beeson, M. MacMahon, and F. Savelli. 2004. Local metrical
and global topological maps in the hybrid spatial semantic hierarchy. In \textit{Proceedings
of the International Conference on Robotics and Automation (ICRA)}.

Kushmerick, N., S. Hanks, and D.S. Weld. 1995. An algorithm for probabilistic planning.
\textit{Artificial Intelligence} 76:239–286.

Kwok, C.T., D. Fox, and M. Meila. 2004. Real-time particle filters. \textit{Proceedings of the
IEEE} 92:469 – 484. Special Issue on Sequential State Estimation.

Latombe, J.-C. 1991. \textit{Robot Motion Planning.} Boston, MA: Kluwer Academic Publishers.

LaValle, S.M., H. Gonzalez-Banos, C. Becker, and J.C. Latombe. 1997. Motion strategies
for maintaining visibility of a moving target. In \textit{Proceedings of the International
Conference on Robotics and Automation (ICRA)}.

Lawler, E.L., and D.E.Wood. 1966. Branch-and-bound methods: A survey. \textit{Operations
Research} 14:699–719.

Lebeltel, O., P. Bessière, J. Diard, and E. Mazer. 2004. Bayesian robot programming.
\textit{Autonomous Robots} 16:49–97.

Lee, D., and M. Recce. 1997. Quantitative evaluation of the exploration strategies of
a mobile robot. \textit{International Journal of Robotics Research} 16:413–447.

Lenser, S., and M. Veloso. 2000. Sensor resetting localization for poorly modelled mobile
robots. In \textit{Proceedings of the International Conference on Robotics and Automation
(ICRA)}.

Leonard, J.J., and H.F. Durrant-Whyte. 1991. Mobile robot localization by tracking
geometric beacons. \textit{IEEE Transactions on Robotics and Automation} 7:376–382.

Leonard, J.J., and H.J.S. Feder. 1999. A computationally efficient method for largescale
concurrent mapping and localization. In J. Hollerbach and D. Koditschek
(eds.), \textit{Proceedings of the Ninth International Symposium on Robotics Research}, Salt Lake
City, Utah.

Leonard, J.J., and H.J.S. Feder. 2001. Decoupled stochastic mapping. \textit{IEEE Journal of
Ocean Engineering} 26:561–571.

Leonard, J., and P. Newman. 2003. Consistent, convergent, and constant-time SLAM.
In \textit{Proceedings of the IJCAI Workshop on Reasoning with Uncertainty in Robot Navigation},
Acapulco, Mexico.

Leonard, J.J., R.J. Rikoski, P.M. Newman, and M. Bosse. 2002a. Mapping partially
observable features from multiple uncertain vantage points. \textit{International Journal of
Robotics Research} 21:943–975.

Leonard, J., J.D. Tardós, S. Thrun, and H. Choset (eds.). 2002b. \textit{Workshop Notes of the
ICRAWorkshop on Concurrent Mapping and Localization for Autonomous Mobile Robots
(W4)}. Washington, DC: ICRA Conference.

Levenberg, K. 1944. A method for the solution of certain problems in least squares.
\textit{Quarterly Applied Mathematics} 2:164–168.

Li, R., F. Ma, F. Xu, L. Matthies, C. Olson, and Y. Xiong. 2000. Large scale mars
mapping and rover localization using descent and rover imagery. In \textit{Proceedings of
the ISPRS 19th Congress, IAPRS Vol. XXXIII,} Amsterdam.

Likhachev, M., G. Gordon, and S. Thrun. 2004. Planning for Markov decision processes
with sparse stochasticity. In L. Saul, Y. Weiss, and L. Bottou (eds.), \textit{Proceedings
of Conference on Neural Information Processing Systems (NIPS)}. Cambridge, MA:
MIT Press.

Lin, L.-J., and T.M. Mitchell. 1992. Memory approaches to reinforcement learning
in non-Markovian domains. Technical Report CMU-CS-92-138, Carnegie Mellon
University, Pittsburgh, PA.

Littman, M.L., A.R. Cassandra, and L.P. Kaelbling. 1995. Learning policies for partially
observable environments: Scaling up. In A. Prieditis and S. Russell (eds.),
\textit{Proceedings of the Twelfth International Conference on Machine Learning.}

Littman, M.L., R.S. Sutton, and S. Singh. 2001. Predictive representations of state. In
\textit{Advances in Neural Information Processing Systems 14.}

Liu, J., and R. Chen. 1998. Sequential Monte Carlo methods for dynamic systems.
\textit{Journal of the American Statistical Association} 93:1032–1044.

Liu, Y., and S. Thrun. 2003. Results for outdoor-SLAM using sparse extended information
filters. In \textit{Proceedings of the International Conference on Robotics and Automation
(ICRA)}.

Lovejoy,W.S. 1991. A survey of algorithmic methods for partially observable Markov
decision processes. \textit{Annals of Operations Research} 28:47–65.

Lozano-Perez, T. 1983. Spatial planning: A configuration space approach. \textit{IEEE
Transactions on Computers} pp. 108–120.

Lu, F., and E. Milios. 1994. Robot pose estimation in unknown environments by
matching 2d range scans. In \textit{IEEE Computer Vision and Pattern Recognition Conference
(CVPR)}.

Lu, F., and E. Milios. 1997. Globally consistent range scan alignment for environment
mapping. \textit{Autonomous Robots} 4:333–349.

Lu, F., and E. Milios. 1998. Robot pose estimation in unknown environments by
matching 2d range scans. \textit{Journal of Intelligent and Robotic Systems} 18:249–275.

Lumelsky, S., S. Mukhopadhyay, and K. Sun. 1990. Dynamic path planning in sensorbased
terrain acquisition. \textit{IEEE Transactions on Robotics and Automation} 6:462–472.

MacDonald, I.L., andW. Zucchini. 1997. \textit{Hidden Markov and Other Models for Discrete-
Valued Time Series. }London, UK: Chapman and Hall.

Madhavan, R., G. Dissanayake, H. Durrant-Whyte, J. Roberts, P. Corke, and J. Cunningham.
1999. Issues in autonomous navigation of underground vehicles. \textit{Journal
of Mineral Resources Engineering} 8:313–324.

Mahadevan, S., and L. Kaelbling. 1996. The NSF workshop on reinforcement learning:
Summary and observations. \textit{AI Magazine Winter}:89–97.

Mahadevan, S., and N. Khaleeli. 1999. Robust mobile robot navigation using
partially-observable semi-Markov decision processes. Internal report.

Makarenko, A.A., S.B.Williams, F. Bourgoult, and F. Durrant-Whyte. 2002. An experiment
in integrated exploration. In \textit{Proceedings of the IEEE/RSJ Int. Conf. on Intelligent
Robots and Systems (IROS).}

Marquardt, D. 1963. An algorithm for least-squares estimation of nonlinear parameters.
\textit{SIAM Journal of Applied Mathematics} 11:431–441.

Mason, M.T. 2001. \textit{Mechanics of Robotic Manipulation.} Cambridge, MA: MIT Press.

Mataric, M.J. 1990. A distributed model for mobile robot environment-learning and
navigation. Master’s thesis, MIT, Cambridge, MA. Also available as MIT Artificial
Intelligence Laboratory Tech Report AITR-1228.

Matthies, L., E. Gat, R. Harrison, B. Wilcox, R. Volpe, and T. Litwin. 1995. Mars
microrover navigation: Performance evaluation and enhancement. \textit{Autonomous
Robots} 2:291–311.

Maybeck, P.S. 1990. The Kalman filter: An introduction to concepts. In I.J. Cox and
G.T. Wilfong (eds.), \textit{Autonomous Robot Vehicles}. Springer Verlag.

Metropolis, N., and S. Ulam. 1949. The Monte Carlo method. \textit{Journal of the American
Statistical Association} 44:335–341.

Mikhail, E. M., J. S. Bethel, and J. C. McGlone. 2001. \textit{Introduction to Modern Photogrammetry.}
John Wiley and Sons, Inc.

Mine, H., and S. Osaki. 1970. \textit{Markovian Decision Processes}. American Elsevier.

Minka, T. 2001. \textit{A family of algorithms for approximate Bayesian inference.} PhD thesis,
MIT Media Lab, Cambridge, MA.

Monahan, G.E. 1982. A survey of partially observable Markov decision processes:
Theory, models, and algorithms. \textit{Management Science }28:1–16.

Montemerlo, M., N. Roy, and S. Thrun. 2003a. Perspectives on standardization in
mobile robot programming: The Carnegie Mellon navigation (CARMEN) toolkit.
In \textit{Proceedings of the Conference on Intelligent Robots and Systems (IROS)}. Software
package for download at www.cs.cmu.edu/carmen.

Montemerlo, M., and S. Thrun. 2004. Large-scale robotic 3-d mapping of urban structures.
In \textit{Proceedings of the International Symposium on Experimental Robotics (ISER)},
Singapore. Springer Tracts in Advanced Robotics (STAR).

Montemerlo, M., S. Thrun, D. Koller, and B. Wegbreit. 2002a. FastSLAM: A factored
solution to the simultaneous localization and mapping problem. In \textit{Proceedings of
the AAAI National Conference on Artificial Intelligence,} Edmonton, Canada. AAAI.

Montemerlo, M., S. Thrun, D. Koller, and B. Wegbreit. 2003b. FastSLAM 2.0: An
improved particle filtering algorithm for simultaneous localization and mapping
that provably converges. In \textit{Proceedings of the Sixteenth International Joint Conference
on Artificial Intelligence (IJCAI)}, Acapulco, Mexico. IJCAI.

Montemerlo, M., W. Whittaker, and S. Thrun. 2002b. Conditional particle filters for
simultaneous mobile robot localization and people-tracking. In \textit{Proceedings of the
International Conference on Robotics and Automation (ICRA).}

Moore, A.W. 1991. Variable resolution dynamic programming: Efficiently learning
action maps in multivariate real-valued state-spaces. In \textit{Proceedings of the Eighth
International Workshop on Machine Learning,} pp. 333–337.

Moravec, H.P. 1988. Sensor fusion in certainty grids for mobile robots. \textit{AI Magazine}
9:61–74.

Moravec, H.P., and M.C. Martin, 1994. Robot navigation by 3D spatial evidence grids.
Mobile Robot Laboratory, Robotics Institute, Carnegie Mellon University.

Moutarlier, P., and R. Chatila. 1989a. An experimental system for incremental environment
modeling by an autonomous mobile robot. In \textit{1st International Symposium
on Experimental Robotics,} Montreal.

Moutarlier, P., and R. Chatila. 1989b. Stochastic multisensory data fusion for mobile
robot location and environment modeling. In \textit{5th Int. Symposium on Robotics
Research,} Tokyo.

Mozer,M.C., and J.R. Bachrach. 1989. Discovering the structure of a reactive environment
by exploration. Technical Report CU-CS-451-89, Dept. of Computer Science,
University of Colorado, Boulder.

Murphy, K. 2000a. Bayesian map learning in dynamic environments. In \textit{Advances in
Neural Information Processing Systems (NIPS)}. Cambridge, MA: MIT Press.

Murphy, K. 2000b. A survey of POMDP solution techniques. Technical report, UC
Berkeley, Berkeley, CA.

Murphy, K., and S. Russell. 2001. Rao-Blackwellized particle filtering for dynamic
Bayesian networks. In A. Doucet, N. de Freitas, and N. Gordon (eds.), \textit{Sequential
Monte Carlo Methods in Practice}, pp. 499–516. Springer Verlag.

Murphy, R. 2000c. \textit{Introduction to AI Robotics.} Cambridge, MA: MIT Press.

Murphy, R. 2004. Human-robot interaction in rescue robotics. \textit{IEEE Systems, Man and
Cybernetics Part C: Applications and Reviews} 34.

Narendra, P.M., and K. Fukunaga. 1977. A branch and bound algorithm for feature
subset selection. \textit{IEEE Transactions on Computers} 26:914–922.

Neira, J., M.I. Ribeiro, and J.D. Tardós. 1997. Mobile robot localisation and map
building using monocular vision. In \textit{Proceedings of the International Symposium On
Intelligent Robotics Systems,} Stockholm, Sweden.

Neira, J., and J.D. Tardós. 2001. Data association in stochastic mapping using the joint
compatibility test. \textit{IEEE Transactions on Robotics and Automation} 17:890–897.

Neira, J., J.D. Tardós, and J.A. Castellanos. 2003. Linear time vehicle relocation
in SLAM. In \textit{Proceedings of the International Conference on Robotics and Automation
(ICRA)}.

Nettleton, E.W., P.W. Gibbens, and H.F. Durrant-Whyte. 2000. Closed form solutions
to the multiple platform simultaneous localisation and map building (slam)
problem. In Bulur V. Dasarathy (ed.),\textit{ Sensor Fusion: Architectures, Algorithms, and
Applications IV}, volume 4051, pp. 428–437, Bellingham.

Nettleton, E., S. Thrun, and H. Durrant-Whyte. 2003. Decentralised slam with lowbandwidth
communication for teams of airborne vehicles. In \textit{Proceedings of the
International Conference on Field and Service Robotics, }Lake Yamanaka, Japan.

Newman, P. 2000. \textit{On the Structure and Solution of the Simultaneous Localisation and
Map Building Problem.} PhD thesis, Australian Centre for Field Robotics, University
of Sydney, Sydney, Australia.

Newman, P.,M. Bosse, and J. Leonard. 2003. Autonomous feature-based exploration.
In \textit{Proceedings of the International Conference on Robotics and Automation (ICRA)}.

Newman, P.M., and H.F. Durrant-Whyte. 2001. A new solution to the simultaneous
and map building (SLAM) problem. In \textit{Proceedings of SPIE.}

Newman, P., and J.L. Rikoski. 2003. Towards constant-time slam on an autonomous
underwater vehicle using synthetic aperture sonar. In \textit{Proceedings of the International
Symposium of Robotics Research,} Sienna, Italy.

Neyman, J. 1934. On the two different aspects of the representative model: the
method of stratified sampling and the method of purposive selection. \textit{Journal of
the Royal Statistical Society} 97:558–606.

Ng, A.Y., A. Coates, M. Diel, V. Ganapathi, J. Schulte, B. Tse, E. Berger, and E. Liang.
2004. Autonomous inverted helicopter flight via reinforcement learning. In \textit{Proceedings
of the International Symposium on Experimental Robotics (ISER)}, Singapore.
Springer Tracts in Advanced Robotics (STAR).

Ng, A.Y., and M. Jordan. 2000. PEGASUS: a policy search method for large MDPs
and POMDPs. In \textit{Proceedings of Uncertainty in Artificial Intelligence.}

Ng, A.Y., J. Kim, M.I. Jordan, and S. Sastry. 2003. Autonomous helicopter flight via
reinforcement learning. In S. Thrun, L. Saul, and B. Schölkopf (eds.), \textit{Proceedings
of Conference on Neural Information Processing Systems (NIPS)}. Cambridge, MA: MIT
Press.

Nieto, J., J.E. Guivant, and E.M. Nebot. 2004. The hybrid metric maps (HYMMs):
A novel map representation for dense SLAM. In \textit{Proceedings of the International
Conference on Robotics and Automation (ICRA)}.

Nilsson, N.J. 1982. \textit{Principles of Artificial Intelligence.} Berlin, New York: Springer
Publisher.

Nilsson, N. 1984. Shakey the robot. Technical Report 323, SRI International, Menlo
Park, CA.

Nourbakhsh, I. 1987. \textit{Interleaving Planning and Execution for Autonomous Robots.}
Boston, MA: Kluwer Academic Publishers.

Nourbakhsh, I., R. Powers, and S. Birchfield. 1995. DERVISH an office-navigating
robot. \textit{AI Magazine }16.

Nüchter, A., H. Surmann, K. Lingemann, J. Hertzberg, and S. Thrun. 2004. 6D SLAM
with application in autonomous mine mapping. In \textit{Proceedings of the International
Conference on Robotics and Automation (ICRA)}.

Oore, S., G.E. Hinton, and G. Dudek. 1997. A mobile robot that learns its place. \textit{Neural
Computation} 9:683–699.

Ortin, D., J. Neira, and J.M. Montiel. 2004. Relocation using laser and vision. In
\textit{Proceedings of the International Conference on Robotics and Automation (ICRA)}, New
Orleans.

Park, S., F. Pfenning, and S. Thrun. 2005. A probabilistic progamming language
based upon sampling functions. In \textit{Proceedings of the ACM Symposium on Principles
of Programming Languages (POPL)}, Long Beach, CA. ACM SIGPLAN - SIGACT.

Parr, R., and S. Russell. 1998. Reinforcement learning with hierarchies of machines.
In \textit{Advances in Neural Information Processing Systems 10}. Cambridge, MA: MIT Press.

Paskin, M.A. 2003. Thin junction tree filters for simultaneous localization and mapping.
In \textit{Proceedings of the Sixteenth International Joint Conference on Artificial Intelligence
(IJCAI)}, Acapulco, Mexico. IJCAI.

Paul, R.P. 1981. \textit{Robot Manipulators: Mathematics, Programming, and Control.} Cambridge,
MA: MIT Press.

Pearl, J. 1988. \textit{Probabilistic reasoning in intelligent systems: networks of plausible inference.}
San Mateo, CA: Morgan Kaufmann.

Pierce, D., and B. Kuipers. 1994. Learning to explore and build maps. In \textit{Proceedings of
the Twelfth National Conference on Artificial Intelligence,} pp. 1264–1271, Menlo Park.
AAAI, AAAI Press/MIT Press.

Pineau, J., G. Gordon, and S. Thrun. 2003a. Applying metric trees to belief-point
POMDPs. In S. Thrun, L. Saul, and B. Schölkopf (eds.), \textit{Proceedings of Conference on
Neural Information Processing Systems (NIPS).} Cambridge, MA: MIT Press.

Pineau, J., G. Gordon, and S. Thrun. 2003b. Point-based value iteration: An anytime
algorithm for POMDPs. In \textit{Proceedings of the Sixteenth International Joint Conference
on Artificial Intelligence (IJCAI),} Acapulco, Mexico. IJCAI.

Pineau, J., G. Gordon, and S. Thrun. 2003c. Policy-contingent abstraction for robust
robot control. In \textit{Proceedings of the Conference on Uncertainty in AI (UAI)}, Acapulco,
Mexico.

Pineau, J., M. Montemerlo, N. Roy, S. Thrun, and M. Pollack. 2003d. Towards robotic
assistants in nursing homes: challenges and results. \textit{Robotics and Autonomous Systems}
42:271–281.

Pitt, M., and N. Shephard. 1999. Filtering via simulation: auxiliary particle filter.
\textit{Journal of the American Statistical Association} 94:590–599.

Poon, K.-M. 2001. A fast heuristic algorithm for decision-theoretic planning. Master’s
thesis, The Hong Kong University of Science and Technology.

Poupart, P., and C. Boutilier. 2000. Value-directed belief state approximation for
POMDPs. In \textit{Proceedings of the Conference on Uncertainty in AI (UAI)}, pp. 279–288.

Poupart, P., L.E. Ortiz, and C. Boutilier. 2001. Value-directed sampling methods for
monitoring POMDPs. In \textit{Proceedings of the 17th Annual Conference on Uncertainty in
AI (UAI)}.

Procopiuc, O., P.K. Agarwal, L. Arge, and J.S. Vitter. 2003. Bkd-tree: A dynamic scalable
kd-tree. In T. Hadzilacos, Y. Manolopoulos, J.F. Roddick, and Y. Theodoridis
(eds.), \textit{Advances in Spatial and Temporal Databases,} Santorini Island, Greece. Springer
Verlag.

Rabiner, L.R., and B.H. Juang. 1986. An introduction to hidden Markov models. \textit{IEEE
ASSP Magazine} 3:4–16.

Raibert, M.H. 1991. Trotting, pacing, and bounding by a quadruped robot. \textit{Journal of
Biomechanics} 23:79–98.

Raibert, M.H., M. Chepponis, and H.B. Brown Jr. 1986. Running on four legs as
though they were one. \textit{IEEE Transactions on Robotics and Automation} 2:70–82.

Rao, C.R. 1945. Information and accuracy obtainable in estimation of statistical parameters.
\textit{Bulletin of the Calcutta Mathematical Society }37:81–91.

Rao, N., S. Hareti,W. Shi, and S. Iyengar. 1993. Robot navigation in unknown terrains:
Introductory survey of non-heuristic algorithms. Technical Report ORNL/TM-
12410, Oak Ridge National Laboratory.

Reed, M.K., and P.K. Allen. 1997. A robotic system for 3-d model acquisition from
multiple range images. In \textit{Proceedings of the International Conference on Robotics and
Automation (ICRA)}.

Rees,W.G. 2001. \textit{Physical Principles of Remote Sensing (Topics in Remote Sensing).} Cambridge,
UK: Cambridge University Press.

Reif, J.H. 1979. Complexity of the mover’s problem and generalizations. In \textit{Proceedings
of the 20th IEEE Symposium on Foundations of Computer Science,} pp. 421–427.

Rekleitis, I.M., G. Dudek, and E.E. Milios. 2001a. Multi-robot collaboration for robust
exploration. \textit{Annals of Mathematics and Artificial Intelligence} 31:7–40.

Rekleitis, I., R. Sim, G. Dudek, and E. Milios. 2001b. Collaborative exploration for
map construction. In \textit{IEEE International Symposium on Computational Intelligence in
Robotics and Automation.}

Rencken, W.D. 1993. Concurrent localisation and map building for mobile robots
using ultrasonic sensors. In \textit{Proceedings of the IEEE/RSJ Int. Conf. on Intelligent Robots
and Systems (IROS)}, pp. 2129–2197.

Reuter, J. 2000. Mobile robot self-localization using PDAB. In \textit{Proceedings of the International
Conference on Robotics and Automation (ICRA}).

Rikoski, R., J. Leonard, P. Newman, and H. Schmidt. 2004. Trajectory sonar perception
in the ligurian sea. In \textit{Proceedings of the International Symposium on Experimental
Robotics (ISER)}, Singapore. Springer Tracts in Advanced Robotics (STAR).

Rimon, E., and D.E. Koditschek. 1992. Exact robot navigation using artificial potential
functions. \textit{IEEE Transactions on Robotics and Automation} 8:501–518.

Rivest, R.L., and R.E. Schapire. 1987a. Diversity-based inference of finite automata.
In \textit{Proceedings of Foundations of Computer Science.}

Rivest, R.L., and R.E. Schapire. 1987b. A new approach to unsupervised learning in
deterministic environments. In P. Langley (ed.), \textit{Proceedings of the Fourth International
Workshop on Machine Learning,} pp. 364–375, Irvine, California.

Robbins, H. 1952. Some aspects of the sequential design of experiments. \textit{Bulletin of
the American Mathemtical Society} 58:529–532.

Rosencrantz, M., G. Gordon, and S. Thrun. 2004. Learning low dimensional predictive
representations. In \textit{Proceedings of the Twenty-First International Conference on
Machine Learning,} Banff, Alberta, Canada.

Roumeliotis, S.I., and G.A. Bekey. 2000. Bayesian estimation and Kalman filtering: A
unified framework for mobile robot localization. In \textit{Proceedings of the International
Conference on Robotics and Automation (ICRA)}, pp. 2985–2992.

Rowat, P.F. 1979. \textit{Representing the Spatial Experience and Solving Spatial problems in a
Simulated Robot Environment}. PhD thesis, University of British Columbia, Vancouver,
BC, Canada.

Roy, B.V., and J.N. Tsitsiklis. 1996. Stable linear approximations to dynamic programming
for stochastic control problems with local transitions. In D. Touretzky,
M. Mozer, and M.E. Hasselmo (eds.), \textit{Advances in Neural Information Processing Systems
8}. Cambridge, MA: MIT Press.

Roy, N., W. Burgard, D. Fox, and S. Thrun. 1999. Coastal navigation: Robot navigation
under uncertainty in dynamic environments. In \textit{Proceedings of the International
Conference on Robotics and Automation (ICRA)}.

Roy, N., and G. Dudek. 2001. Collaborative exploration and rendezvous: Algorithms,
performance bounds and observations. \textit{Autonomous Robots} 11:117–136.

Roy, N., G. Gordon, and S. Thrun. 2004. Finding approximate POMDP solutions
through belief compression. \textit{Journal of Artificial Intelligence Research.} To appear.

Roy, N., J. Pineau, and S. Thrun. 2000. Spoken dialogue management using probabilistic
reasoning. In \textit{Proceedings of the 38th Annual Meeting of the Association for
Computational Linguistics (ACL-2000),} Hong Kong.

Roy, N., and S. Thrun. 2002. Motion planning through policy search. In \textit{Proceedings
of the IEEE/RSJ Int. Conf. on Intelligent Robots and Systems (IROS)}.

Rubin, D.B. 1988. Using the SIR algorithm to simulate posterior distributions. In
M.H. Bernardo, K.M. an DeGroot, D.V. Lindley, and A.F.M. Smith (eds.), \textit{Bayesian
Statistics 3.} Oxford, UK: Oxford University Press.

Rubinstein, R.Y. 1981. \textit{Simulation and the Monte Carlo Method}. John Wiley and Sons,
Inc.

Russell, S., and P. Norvig. 2002. \textit{Artificial Intelligence: A Modern Approach.} Englewood
Cliffs, NJ: Prentice Hall.

Saffiotti, A. 1997. The uses of fuzzy logic in autonomous robot navigation. \textit{Soft
Computing} 1:180–197.

Sahin, E., P. Gaudiano, and R. Wagner. 1998. A comparison of visual looming and
sonar as mobile robot range sensors. In \textit{Proceedings of the Second International Conference
on Cognitive And Neural Systems,} Boston, MA.

Salichs, M.A., J.M. Armingol, L. Moreno, and A. Escalera. 1999. Localization system
for mobile robots in indoor environments. \textit{Integrated Computer-Aided Engineering 6:}
303–318.

Salichs, M.A., and L. Moreno. 2000. Navigation of mobile robots: Open questions.
\textit{Robotica} 18:227–234.

Sandwell, D.T., 1997. Exploring the ocean basins with satellite altimeter data.
http://julius.ngdc.noaa.gov/mgg/bathymetry/predicted/explore.HTML.

Saranli, U., and D.E. Koditschek. 2002. Back flips with a hexapedal robot. In \textit{Proceedings
of the International Conference on Robotics and Automation (ICRA)}, volume 3, pp.
128–134.

Schiele, B., and J.L. Crowley. 1994. A comparison of position estimation techniques
using occupancy grids. In \textit{Proceedings of the International Conference on Robotics and
Automation (ICRA)}.

Schoppers, M.J. 1987. Universal plans for reactive robots in unpredictable environments.
In J. McDermott (ed.), \textit{Proceedings of the Tenth International Joint Conference
on Artificial Intelligence (IJCAI-87)}, pp. 1039–1046, Milan, Italy. Morgan Kaufmann.

Schulz, D., W. Burgard, D. Fox, and A.B. Cremers. 2001a. Tracking multiple moving
objects with a mobile robot. In \textit{Proceedings of the IEEE Computer Society Conference
on Computer Vision and Pattern Recognition (CVPR)}, Kauai, Hawaii.

Schulz, D., W. Burgard, D. Fox, and A.B. Cremers. 2001b. Tracking multiple moving
targets with a mobile robot using particle filters and statistical data association. In
\textit{Proceedings of the International Conference on Robotics and Automation (ICRA)}.

Schulz, D., and D. Fox. 2004. Bayesian color estimation for adaptive vision-based
robot localization. In \textit{Proceedings of the IEEE/RSJ Int. Conf. on Intelligent Robots and
Systems (IROS)}.

Schwartz, J.T., M. Scharir, and J. Hopcroft. 1987. \textit{Planning, Geometry and Complexity of
Robot Motion.} Norwood, NJ: Ablex Publishing Corporation.

Scott, D.W. 1992. \textit{Multivariate density estimation: theory, practice, and visualization.} John
Wiley and Sons, Inc.

Shaffer, G., J. Gonzalez, and A. Stentz. 1992. Comparison of two range-based estimators
for a mobile robot. In \textit{SPIE Conf. on Mobile Robots VII}, pp. 661–667.

Sharma, R. 1992. Locally efficient path planning in an uncertain, dynamic environment
using a probabilistic model. T-RA 8:105–110.

Shatkay, H, and L. Kaelbling. 1997. Learning topological maps with weak local odometric
information. In \textit{Proceedings of IJCAI-97}. IJCAI, Inc.

Siegwart, R., K.O. Arras, S. Bouabdallah, D. Burnier, G. Froidevaux, X. Greppin,
B. Jensen, A. Lorotte, L. Mayor, M. Meisser, R. Philippsen, R. Piguet, G. Ramel,
G. Terrien, and N. Tomatis. 2003. A large scale installation of personal robots.
Special issue on socially interactive robots. \textit{Robotics and Autonomous Systems 42}:
203–222.

Siegwart, R., and I. Nourbakhsh. 2004. \textit{Introduction to Autonomous Mobile Robots.}
Cambridge, MA: MIT Press.

Sim, R., G. Dudek, and N. Roy. 2004. Online control policy optimization for minimizing
map uncertainty during exploration. In \textit{Proceedings of the International Conference
on Robotics and Automation (ICRA)}.

Simmons, R.G., D. Apfelbaum, W. Burgard, D. Fox, M. Moors, S. Thrun, and
H. Younes. 2000a. Coordination for multi-robot exploration and mapping. In
\textit{Proc. of the National Conference on Artificial Intelligence (AAAI)}.

Simmons, R.G., J. Fernandez, R. Goodwin, S. Koenig, and J. O’Sullivan. 2000b.
Lessons learned from Xavier. \textit{IEEE Robotics and Automation Magazine} 7:33–39.

Simmons, R.G., and S. Koenig. 1995. Probabilistic robot navigation in partially observable
environments. In \textit{Proceedings of the International Joint Conference on Artificial
Intelligence (IJCAI).}

Simmons, R.G., S. Thrun, C. Athanassiou, J. Cheng, L. Chrisman, R. Goodwin, G.-T.
Hsu, and H. Wan. 1992. Odysseus: An autonomous mobile robot. \textit{AI Magazine.}
extended abstract.

Singh, K., and K. Fujimura. 1993. Map making by cooperating mobile robots. In
\textit{Proceedings of the International Conference on Robotics and Automation (ICRA)}, pp. 254–
259.

Smallwood, R.W., and E.J. Sondik. 1973. The optimal control of partially observable
Markov processes over a finite horizon. \textit{Operations Research} 21:1071–1088.

Smith, A.F.M., and A.E. Gelfand. 1992. Bayesian statistics without tears: a samplingresampling
perspective. \textit{American Statistician} 46:84–88.

Smith, R.C., and P. Cheeseman. 1986. On the representation and estimation of spatial
uncertainty. \textit{International Journal of Robotics Research} 5:56–68.

Smith, R.,M.
Self, and P. Cheeseman. 1990. Estimating uncertain spatial relationships
in robotics. In I.J. Cox and G.T. Wilfong (eds.), \textit{Autonomous Robot Vehicles}, pp. 167–
193. Springer-Verlag.

Smith, S. M., and S. E. Dunn. 1995. The ocean explorer AUV: A modular platform
for coastal sensor deployment. In \textit{Proceedings of the Autonomous Vehicles in Mine
Countermeasures Symposium}. Naval Postgraduate School.

Smith, T., and R.G. Simmons. 2004. Heuristic search value iteration for POMDPs. In
\textit{Proceedings of the 20th Annual Conference on Uncertainty in AI (UAI)}.

Soatto, S., and R. Brockett. 1998. Optimal structure from motion: Local ambiguities
and global estimates. In \textit{Proceedings of the Conference on Computer Vision and Pattern
Recognition (CVPR)}, pp. 282–288.

Sondik, E. 1971. \textit{The Optimal Control of Partially Observable Markov Processes.} PhD
thesis, Stanford University.

Stachniss, C., and W. Burgard. 2003. Exploring unknown environments with mobile
robots using coverage maps. In \textit{Proceedings of the Sixteenth International Joint
Conference on Artificial Intelligence (IJCAI),} Acapulco, Mexico. IJCAI.

Stachniss, C., and W. Burgard. 2004. Exploration with active loop-closing for Fast-
SLAM. In \textit{Proceedings of the IEEE/RSJ Int. Conf. on Intelligent Robots and Systems
(IROS)}.

Steels, L. 1991. Towards a theory of emergent functionality. In J-A. Meyer and R.Wilson
(eds.), \textit{Simulation of Adaptive Behavior}. Cambridge, MA: MIT Press.

Stentz, A. 1995. The focussed D* algorithm for real-time replanning. In \textit{Proceedings of
IJCAI-95}.

Stewart, B., J. Ko, D. Fox, and K. Konolige. 2003. The revisiting problem in mobile
robot map building: A hierarchical Bayesian approach. In \textit{Proceedings of the
Conference on Uncertainty in AI (UAI)}, Acapulco, Mexico.

Strassen, V. 1969. Gaussian elimination is not optimal. \textit{Numerische Mathematik} 13:
354–356.

Stroupe, A.W. 2004. Value-based action selection for exploration and mapping with
robot teams. In \textit{Proceedings of the International Conference on Robotics and Automation
(ICRA)}.

Sturges, H. 1926. The choice of a class-interval. \textit{Journal of the American Statistical
Association} 21:65–66.

Subrahmaniam, K. 1979. \textit{A Primer In Probability}. New York, NY: M. Dekker.

Swerling, P. 1958. A proposed stagewise differential correction procedure for satellite
tracking and prediction. Technical Report P-1292, RAND Corporation.

Tailor, C.J., and D.J. Kriegman. 1993. Exloration strategies for mobile robots. In
\textit{Proceedings of the International Conference on Robotics and Automation (ICRA)}, pp. 248–
253.

Tanner, M.A. 1996. \textit{Tools for Statistical Inference}. New York: Springer Verlag. 3rd
edition.

Tardós, J.D., J. Neira, P.M. Newman, and J.J. Leonard. 2002. Robust mapping and localization
in indoor environments using sonar data. \textit{International Journal of Robotics
Research} 21:311–330.

Teller, S., M. Antone, Z. Bodnar, M. Bosse, S. Coorg, M. Jethwa, and N. Master. 2001.
Calibrated, registered images of an extended urban area. In \textit{Proceedings of the Conference
on Computer Vision and Pattern Recognition (CVPR)}.

Theocharous, G., K. Rohanimanesh, and S. Mahadevan. 2001. Learning hierarchical
partially observed Markov decision process models for robot navigation. In
\textit{Proceedings of the International Conference on Robotics and Automation (ICRA)}.

Thorp, E.O. 1966. \textit{Elementary Probability}. R.E. Krieger.

Thrun, S. 1992. Efficient exploration in reinforcement learning. Technical Report
CMU-CS-92-102, Carnegie Mellon University, Computer Science Department,
Pittsburgh, PA.
Thrun, S. 1993. Exploration and model building in mobile robot domains. In E. Ruspini
(ed.), \textit{Proceedings of the IEEE International Conference on Neural Networks}, pp.
175–180, San Francisco, CA. IEEE Neural Network Council.

Thrun, S. 1998a. Bayesian landmark learning for mobile robot localization. \textit{Machine
Learning }33.

Thrun, S. 1998b. Learning metric-topological maps for indoor mobile robot navigation.
\textit{Artificial Intelligence} 99:21–71.

Thrun, S. 2000a. Monte Carlo POMDPs. In S.A. Solla, T.K. Leen, and K.-R. Müller
(eds.), \textit{Advances in Neural Information Processing Systems 12}, pp. 1064–1070. Cambridge,
MA: MIT Press.

Thrun, S. 2000b. Towards programming tools for robots that integrate probabilistic
computation and learning. In \textit{Proceedings of the IEEE International Conference on
Robotics and Automation (ICRA)}, San Francisco, CA. IEEE.

Thrun, S. 2001. A probabilistic online mapping algorithm for teams of mobile robots.
\textit{International Journal of Robotics Research} 20:335–363.

Thrun, S. 2002. Robotic mapping: A survey. In G. Lakemeyer and B. Nebel (eds.),
\textit{Exploring Artificial Intelligence in the New Millenium.} Morgan Kaufmann.

Thrun, S. 2003. Learning occupancy grids with forward sensor models. \textit{Autonomous
Robots }15:111–127.

Thrun, S., M. Beetz, M. Bennewitz, W. Burgard, A.B. Cremers, F. Dellaert, D. Fox,
D. Hähnel, C. Rosenberg, N. Roy, J. Schulte, and D. Schulz. 2000a. Probabilistic
algorithms and the interactive museum tour-guide robot minerva. \textit{International
Journal of Robotics Research} 19:972–999.

Thrun, S., A. Bücken, W. Burgard, D. Fox, T. Fröhlinghaus, D. Henning, T. Hofmann,
M. Krell, and T. Schmidt. 1998a. Map learning and high-speed navigation in
RHINO. In D. Kortenkamp, R.P. Bonasso, and R. Murphy (eds.), \textit{AI-based Mobile
Robots: Case Studies of Successful Robot Systems}, pp. 21–52. Cambridge, MA: MIT
Press.

Thrun, S., W. Burgard, and D. Fox. 2000b. A real-time algorithm for mobile robot
mapping with applications to multi-robot and 3D mapping. In \textit{Proceedings of the
International Conference on Robotics and Automation (ICRA).}

Thrun, S., M. Diel, and D. Hähnel. 2003. Scan alignment and 3d surface modeling
with a helicopter platform. In \textit{Proceedings of the International Conference on Field and
Service Robotics,} Lake Yamanaka, Japan.

Thrun, S., D. Fox, and W. Burgard. 1998b. A probabilistic approach to concurrent
mapping and localization for mobile robots. \textit{Machine Learning} 31:29–53. Also appeared
in Autonomous Robots 5, 253–271 (joint issue).

Thrun, S., D. Fox, and W. Burgard. 2000c. Monte Carlo localization with mixture
proposal distribution. In \textit{Proceedings of the AAAI National Conference on Artificial
Intelligence,} Austin, TX. AAAI.

Thrun, S., D. Koller, Z. Ghahramani, H. Durrant-Whyte, and A.Y. Ng. 2002. Simultaneous
mapping and localization with sparse extended information filters. In J.-D.
Boissonnat, J. Burdick, K. Goldberg, and S. Hutchinson (eds.),\textit{ Proceedings of the
Fifth International Workshop on Algorithmic Foundations of Robotics,} Nice, France.

Thrun, S., and Y. Liu. 2003. Multi-robot SLAM with sparse extended information filers.
In \textit{Proceedings of the 11th International Symposium of Robotics Research (ISRR’03)},
Sienna, Italy. Springer.

Thrun, S., Y. Liu, D. Koller, A.Y. Ng, Z. Ghahramani, and H. Durrant-Whyte. 2004a.
Simultaneous localization and mapping with sparse extended information filters.
\textit{International Journal of Robotics Research} 23.

Thrun, S., C. Martin, Y. Liu, D. Hähnel, R. Emery-Montemerlo, D. Chakrabarti, and
W. Burgard. 2004b. A real-time expectation maximization algorithm for acquiring
multi-planar maps of indoor environments with mobile robots. \textit{IEEE Transactions
on Robotics} 20:433–443.

Thrun, S., S. Thayer, W. Whittaker, C. Baker, W. Burgard, D. Ferguson, D. Hähnel,
M. Montemerlo, A. Morris, Z. Omohundro, C. Reverte, and W. Whittaker. 2004c.
Autonomous exploration and mapping of abandoned mines. \textit{IEEE Robotics and
Automation Magazine.} Forthcoming.

Tomasi, C., and T. Kanade. 1992. Shape and motion from image streams under
orthography: A factorization method. \textit{International Journal of Computer Vision 9:}
137–154.

Tomatis, N., I. Nourbakhsh, and R. Siegwart. 2002. Hybrid simultaneous localization
and map building: closing the loop with multi-hypothesis tracking. In \textit{Proceedings
of the International Conference on Robotics and Automation (ICRA)}.

Uhlmann, J., M. Lanzagorta, and S. Julier. 1999. The NASA mars rover: A testbed for
evaluating applications of covariance intersection. In \textit{Proceedings of the SPIE 13th
Annual Symposium in Aerospace/Defence Sensing, Simulation and Controls}.

United Nations, and International Federation of Robotics. 2004. \textit{World Robotics 2004}.
New York and Geneva: United Nations.

Urmson, C., B. Shamah, J. Teza, M.D. Wagner, D. Apostolopoulos, and W.R. Whittaker.
2001. A sensor arm for robotic antarctic meteorite search. In \textit{Proceedings of
the 3rd International Conference on Field and Service Robotics,} Helsinki, Finland.

Vaganay, J., J. Leonard, J.A. Curcio, and J.S.Willcox. 2004. Experimental validation of
the moving long base-line navigation concept. In \textit{Proceedings of the IEEE Conference
on Autonomous Underwater Vehicles.}

van der Merwe, R. 2004. \textit{Sigma-Point Kalman Filters for Probabilistic Inference in Dynamic
State-Space Models}. PhD thesis, OGI School of Science \& Engineering.

van der Merwe, R., N. de Freitas, A. Doucet, and E.Wan. 2001. The unscented particle
filter. In \textit{Advances in Neural Information Processing Systems 13}.

Vlassis, N., B. Terwijn, and B. Kröse. 2002. Auxiliary particle filter robot localization
from high-dimensional sensor observations. In \textit{Proceedings of the International
Conference on Robotics and Automation (ICRA)}.

Vukobratovic, M. 1989. \textit{Introduction to Robotics}. Berlin, New York: Springer Publisher.

Wang, C.-C., C. Thorpe, and S. Thrun. 2003. Online simultaneous localization and
mapping with detection and tracking of moving objects: Theory and results from a
ground vehicle in crowded urban areas. In \textit{Proceedings of the International Conference
on Robotics and Automation (ICRA)}.

Washington, R. 1997. BI-POMDP: Bounded, incremental, partially-observable
Markov-model planning. In \textit{Proceedings of the European Conference on Planning
(ECP)}, Toulouse, France.

Watkins, C.J.C.H. 1989. \textit{Learning from Delayed Rewards.} PhD thesis, King’s College,
Cambridge, England.

Weiss, G., C. Wetzler, and E. von Puttkamer. 1994. Keeping track of position and
orientation of moving indoor systems by correlation of range-finder scans. In \textit{Proceedings
of the International Conference on Intelligent Robots and Systems}, pp. 595–601.

Wesley, M.A., and T. Lozano-Perez. 1979. An algorithm for planning collision-free
paths among polyhedral objects. \textit{Communications of the ACM} 22:560–570.

West, M., and P.J. Harrison. 1997. \textit{Bayesian Forecasting and Dynamic Models}, 2nd edition.
New York: Springer-Verlag.

Wettergreen, D., D. Bapna, M. Maimone, and H. Thomas. 1999. Developing Nomad
for robotic exploration of the Atacama Desert. \textit{Robotics and Autonomous Systems} 26:
127–148.

Whaite, P., and F.P. Ferrie. 1997. Autonomous exploration: Driven by uncertainty.
\textit{IEEE Transactions on Pattern Analysis and Machine Intelligence} 19:193–205.

Whitcomb, L. 2000. Underwater robotics: out of the research laboratory and into the
field. In \textit{Proceedings of the International Conference on Robotics and Automation (ICRA)},
pp. 85–90.

Williams, R.J. 1992. Simple statistical gradient-following algorithms for connectionist
reinforcement learning. \textit{Machine Learning} 8:229–256.

Williams, S.B. 2001. \textit{Efficient Solutions to Autonomous Mapping and Navigation Problems.}
PhD thesis, ACFR, University of Sydney, Sydney, Australia.

Williams, S.B., G. Dissanayake, and H.F. Durrant-Whyte. 2001. Constrained initialization
of the simultaneous localization and mapping algorithm. In \textit{Proceedings of
the Symposium on Field and Service Robotics.} Springer Verlag.

Williams, S.B., G. Dissanayake, and H. Durrant-Whyte. 2002. An efficient approach
to the simultaneous localisation and mapping problem. In \textit{Proceedings of the International
Conference on Robotics and Automation (ICRA)}, pp. 406–411.

Winer, B.J., D.R. Brown, and K.M. Michels. 1971. \textit{Statistical Principles in Experimental
Design}. New York: Mc-Graw-Hill.

Winkler, G. 1995. \textit{Image Analysis, Random Fields, and Dynamic Monte Carlo Methods.}
Berlin: Springer Verlag.

Wolf, D.F., and G.S. Sukhatme. 2004. Mobile robot simultaneous localization and
mapping in dynamic environments. \textit{Autonomous Robots.}

Wolf, J., W. Burgard, and H. Burkhardt. 2005. Robust vision-based localization by
combining an image retrieval system with Monte Carlo localization. \textit{IEEE Transactions
on Robotics and Automation.}

Wong, J. 1989. \textit{Terramechanics and Off-Road Vehicles}. Elsevier.

Yamauchi, B., and P. Langley. 1997. Place recognition in dynamic environments.
\textit{Journal of Robotic Systems} 14:107–120.

Yamauchi, B., A. Schultz, and W. Adams. 1999. Integrating exploration and localization
for mobile robots. \textit{Adaptive Systems 7.}

Yoshikawa, T. 1990. \textit{Foundations of Robotics: Analysis and Control}. Cambridge, MA:
MIT Press.

Zhang, N.L., and W. Zhang. 2001. Speeding up the convergence of value iteration
in partially observable Markov decision processes. \textit{Journal of Artificial Intelligence
Research }14:29–51.

Zhao, H., and R. Shibasaki. 2001. A vehicle-borne system of generating textured
CAD model of urban environment using laser range scanner and line cameras. In
\textit{Proc. InternationalWorkshop on Computer Vision Systems (ICVS)}, Vancouver, Canada.

Zlot, R., A.T. Stenz, M.B. Dias, and S. Thayer. 2002. Multi-robot exploration controlled
by a market economy. In \textit{Proceedings of the International Conference on Robotics and
Automation (ICRA)}.


 
\end{document}